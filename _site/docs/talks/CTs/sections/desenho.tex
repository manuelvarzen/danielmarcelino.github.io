
\begin{itemize}
	\item A amostra é obtida por amostragem estratificada simples, tendo por objetivo estimar proporções para a população de CTs das informações de interesse, controladas pela localização geográfica dessas unidades.
	\item Os estratos naturais são especificados a partir do cruzamento da macrorregião e da categoria REGIC em que está localizada a CT.
	\item Os estratos finais são definidos de acordo com a condição e financiamento pela SENAD. O estrato final certo (com probabilidade de seleção igual a 1) é formado pelas CTs que recebem financiamento da Secretaria; o estrato final amostrado é formado pelas CTs presentes no cadastro, mas que não são financiadas pela Secretaria. Dessa forma, a amostra é estratificada da seguinte maneira.
	
\end{itemize}



\begin{frame}{Referências}

Balinski ML, Young HP (2001) Fair representation: meeting the ideal of one man, one vote. 2nd ed. Brookings Institution Press, Washington DC

Dalton H, (1920) Measurement of the inequality of income. Economic Journal 30: 348-361

Duncan OD, Duncan B (1955) A methodical analysis of segregation indexes. American Sociological Review 20: 210-217

Gallagher M, (1991) Proportionality, Disproportionality and Electoral Systems. Electoral Studies 10: 33-51

Lijphart A, (1994) Electoral Systems and Party Systems. Oxford, New York and Oxford

Loosemore J, Hanby V, (1971) The theoretical limits of maximum distortion: some analytical expressions for electoral systems. British Journal of Political Science 1: 467-477

Monroe BL, (1994) Disproportionality and malapportionments: measuring electoral inequity Electoral Studies. 13: 132-149

Pedersen MN, (1979) The dynamics of European party systems: changing patterns of electoral volatility European Journal of Political Research, 7/1: 1-26

Rae DW, (1967) The political consequences of electoral laws, Yale, New Haven.
Taagepera R Grofman B, (2003) Mapping the indices of seats-votes disproportionality and inter-election volatility, Party Politics, 9(6): 659-677


\end{frame}
